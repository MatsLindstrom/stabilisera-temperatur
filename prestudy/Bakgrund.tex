Energi flödar hela tiden in och ut ur fastigheter, bland annat genom människors kroppsvärme, VVA (värme, ventilation och avlopp) och vädret ute. Dessa energiflöden kan delas in i konstanta energiflöden samt variabla energiflöden. Den främsta variabla energikällan är troligen vädret. Vädret kan, genom sina skiftningar, både ge och ta energi från byggnaden. För att bibehålla en jämn inomhustemperatur i fastigheten kan inte en konstant mängd energi tillföras av värmesystemet, utan energitillförseln måste hela tiden regleras utefter både konstanta samt variabla energiflöden.

I dagsläget regleras de flesta energisystem i fastigheter endast med tanke på utomhustemperaturen i varje ögonblick och på så sätt blir det alltid en fördröjning i uppvärmningen vilket i vårt fall leder till ojämn inomhustemperatur och eventuellt också till onödig energiåtgång.

Vår uppdragsgivare sköter utrustning för uppvärmning av fastigheten. Han har ett pågående projekt med syfte att minska energiförbrukningen i fastigheten samtidigt som ett behagligt inomhusklimat bibehålls. Inom ramen för detta så har en väderstation installerats på taket till fastigheten och sensorer av diverse slag har anslutits på strategiska platser.  Dessa enheter tillåter uppvärmningssystemet att anpassa energianvändningen efter väderlek.

Denna typ av effektivisering av energianvändningen i en fastighet är idag högaktuell på grund av höga energipriser och ökad förståelse för hur vår energianvändning kan påverka planeten negativt.