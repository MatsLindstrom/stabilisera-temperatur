Till att börja med skall vi utföra en litteraturstudie för att se relevant forskning på området energiförluster i en byggnad. Detta för att ge oss banor att fundera i när vi fortsätter att räkna på byggnaden teoretiskt. Teoretisk beräkning kan genomföras genom att till exempel sätta upp en datormodell med de relevanta differentialekvationerna samt parametrarna och approximera lösningen till dessa med exempelvis finita elementmetoden. Här planerar vi att beräkna värmeflöden från de olika väggarna, från taket, från fönster samt från grunden av byggnaden. Vi skall också beräkna hur dessa flöden påverkas av olika väderrelaterade parametrar. Här kan vi säkert få en del hjälp från det tidigare kandidatarbete som behandlat byggnaden.

De framräknade värdena kan med fördel jämföras med empiriska data. Förhoppningsvis kommer vår uppdragsgivare att tillhandahålla oss med mätdata från den i fastigheten installerade mätutrustningen. Dessa data kan även användas för att se hur inomhustemperaturer kovarierar med väder. Denna typ av analys är högintressant för att bygga ett smart självanpassande system som styr inomhusklimatet.

För att se om det finns några defekter på byggnaden som läcker särskilt mycket värme så planerar vi att fotografera byggnaden med hjälp av en värmekamera och på så sätt ha möjlighet att se var de största förlusterna ligger eller om vi har homogena värmeförluster från väggarna.

När vi har fastställt en god modell för hur väder och vind påverkar fastighetens energiflöden så bör vi analysera hur stor energibesparing som kan genomföras genom att ta hänsyn till dessa parametrar i jämförelse med dagens utformning av reglersystem. Här bör vi även ha skapat ett diagram som visar alla värmeflöden in och ut ur fastigheten. Detta kan vara ett gott hjälpmedel för uppdragsgivaren för att fortsätta med det pågående energibesparingsprojektet.